%!TEX root = document.tex

\emph{Social affinity filtering (SAF)} is based on the idea that
affinity between users expressed in social networks via interactions
and activities captured by SAGs is predictive of user preferences.
With social affinity features now defined as in Sec~\ref{ssec:SAfeature}, the task of SAF is
simply one of classification of a user $u$'s preference for a link $i$
\eat{as outlined in Fig~\ref{fig:overview}}. 
Formally, given a user $u$ and item $i$, a SAF classifier is  a
function 
$$f: \x(u,i) \to y\ $$ where $ y \in \{1, 0\}$ and $\x(u,i) = \langle
\cdots,X_{k,u,i},\cdots \rangle$ for all SAG's $k$

Social affinity features $X_{k,u,i}$ captures the fine-grained relationships between user \textit{u}
and other users in $k$th SAG in the context of item \textit{i}.
For example, $k$ could be the SAG of $u$ for the interaction of $\textit{link-like-incoming}$ or the activity of
liking the {\em Obama Re-Election Headquarters} Facebook page.  Then knowing whether
anyone in each SAG $k$ for user $u$ likes item $i$ provides a rich set
of fine-grained features for prediction.

While a classification approach to recommendation might evoke comparisons to standard
\emph{content-based filtering} (CBF)~\cite{newsweeder}, we remark that
CBF is not a social recommendation approach and unlike CBF, SAF does
not require explicit user features or item descriptors; in contrast, SAF
requires only social interactions and learns the affinities between a
user (ego) and the different set of alters as represented by SAGs that
the user interacts or shares common activities with. SAF represent a novel 
approach to \emph{social recommendation} using fine-grained interaction and
activity features that has not been previously proposed in the literature.
 

%SAF represents a
%simple and efficient yet nonetheless novel approach to \emph{social
%  recommendation} using fine-grained interaction and activity features
%that has not been previously proposed in the literature.

%Our task in SAF is to predict whether a given user $u$ likes an item $i$
%or not.  For this purpose, we have the social affinity features $X_{k,u,i}$ defined in 
%Sec~\ref{ssec:SAfeature} based on the various SAGs $k$ defined in 
%Sec~\ref{ssec:sag}; the $X_{k,u,i}$ specifically correspond to features indicating whether any users in the $k$th SAG of
%user $u$ also liked $i$.  For example, $k$ could be the SAG of $u$ for
%the interaction of $\textit{link-like-incoming}$ or the activity of
%liking the {\em Obama Re-Election Headquarters} Facebook page.  Then knowing whether
%anyone in each SAG $k$ for user $u$ likes item $i$ provides a rich set
%of fine-grained features for prediction. It is up to SAF to learn how
%to weight each SAG $k$ to aggregate all SAF preferences into one final
%prediction, which is done by training a classifier on historical data.

  

%To train $f$, one simply provides a dataset of historical observations $D = \{ 
%\x(u,i) \to \likes(u,i) \}$ where $f$ could be a linear
%classifier trained by an SVM, logistic regression, or na\"{i}ve Bayes.
%Then for future predictions, we simply are given a new user $u$ and
%item $i$ to predict for and build the feature vector $\x(u,i)$ and 
%predict $\likes(u,i)$ using the trained $f(\x(u,i))$.

