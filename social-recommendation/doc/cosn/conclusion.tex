%!TEX root = document.tex

This paper presented a Social Affinity Filtering (SAF) as a novel method
for social recommendation based on fine-grained user interactions and
activities in a social network. We evaluated the SAF on a dataset collected 
from a Facebook App we built, showing that SAF yields 6\% absolute improvement
in accuracy with respect to a state-of-the-art social recommendation engine.
%This is an important result given that SAF is built on standard supervised 
%classification techniques unlike more complex matrix factorization approaches
%typically used in social collaborative filtering. 
Furthermore, we quantified the relative importance of interactions and activities in recommendation.
Among many insights, our results show
that video and photo interactions are more predictive than other
modalities, and outgoing interactions are more predictive than
incoming, and that smaller social groups are more predictive than
larger ones. Future directions of research can investigate the nature
of interactions by measuring the level of user engagement -- e.g. if
videos are inherently more engaging, or examine the nature of social
groups via additional metrics -- e.g activity level of the group.

%We proposed Social Affinity Filtering (SAF) as a new method for social
%recommendation.  We first defined social affinity groups (SAGs) of a
%target user by analysing their fine-grained interactions and
%activities.  Then we learned which SAGs were most predictive of the
%target user's preferences leading to SAF.  We evaluated the proposed
%algorithm on a dataset collected from a Facebook App we built, showing
%that SAF yields 6\% absolute improvement in accuracy with respect to a
%state-of-the-art social recommendation engine.  This is an important
%result given that SAF is built on standard supervised classification
%techniques unlike more complex matrix factorization approaches
%typically used in social collaborative filtering.    Among many insights, our results show
%that video and photo interactions are more predictive than other
%modalities, and outgoing interactions are more predictive than
%incoming, and that smaller social groups are more predictive than
%larger ones.  Future directions of research can investigate the nature
%of interactions by measuring the level of user engagement -- e.g. if
%videos are inherently more engaging, or examine the nature of social
%groups via additional metrics -- e.g. the social network within
%members of the group, or activity level of the group.
