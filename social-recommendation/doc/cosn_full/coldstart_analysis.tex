%#suvash#
Collaborative filtering algorithms suffer from the user cold-start problem,
where no historical information about user is available. One of the advantage of SAF is that 
the learning is independent of an individual user. Each social affinity feature for a user-item is
defined by affinity group(ASAG/ISAG) members feedback on given item. 

For user cold-start analysis, we create 10 fold train and test set where each test set
consists of 10\% of total user. Also, user in each train and test set are mutually exclusive.
We hold out 30\% of each user's data from the test set and train two SAF predictors cold-start and non cold-start.
We train Cold-start predictor using training set  and non cold-start predictor by including held out data from test
set to training set. We then evaluate the performance of cold-start and non cold-start predictor on remaining 70% 
of test data. Table ~\ref{tab:coldstart} clearly shows that the accuracy of  SAF predictor for cold-start is comparable
to non cold-start case, which signifies that SAF performs well for cold-start users where most of the
existing methods fail. Hence, unlike standard (social) collaborative filtering , SCF is robust to user 
cold-start problem.  

 
\begin{table}[t!]
\centering
\begin{tabular}{|>{\small}l|>{\small}r|>{\small}r|}
\hline
& \multicolumn{2}{|c|}{\textbf{Accuracy}}\\
\hline
\textbf{Predictor}& \textbf{Cold-Start} & \textbf{Non Cold-Start}\\
\hline
\textbf{Constant} & 0.526  +/-  0.063 & 0.526  +/-  0.063 \\
\hline
\textbf{LR-ISAF} & 0.613 +/- 0.025 & 0.633  +/-  0.045 \\
\hline
\textbf{LR-ASAF(Groups)} & 0.625  +/-  0.025 & 0.651  +/-  0.023 \\
\hline
\textbf{LR-ASAF(Pages)} & 0.601  +/-  0.067 & 0.655  +/-  0.036 \\
\hline
\textbf{LR-ASAF(Favourites)} & 0.616  +/-  0.049 & 0.653  +/-  0.042\\
\hline
\end{tabular}
\caption{Comparision of performance of SAF for user cold-start and non cold-start case}
\label{tab:coldstart}
\end{table}