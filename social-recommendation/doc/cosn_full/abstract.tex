Content recommendation in social networks poses the complex problem of
learning user preferences from a rich and complex set of interactions
(e.g., likes, comments and tags for posts, photos and videos) and
activities (e.g., favourites, group memberships, interests).  While
many social collaborative filtering approaches learn from aggregate statistics over this
social information, we propose a different approach: we first define
social affinity groups (SAGs) of a target user by analysing their
fine-grained interactions (e.g., users who have been tagged in the
target user's video) and activities (e.g., users who have joined the
same special interest group that the target user has joined).  Then we
learn which SAGs are most predictive of the target user's preferences
in a method we term social affinity filtering (SAF).  We apply SAF to
preference data from a set of Facebook users and their
complete interactions with 38,000+ friends collected over a four month
period.  Our analysis demonstrates that SAF yields higher accuracy
than a range of state-of-the-art (social) collaborative filtering approaches and that not all
interactions and activities are equally predictive: among many insights, 
we show certain user-to-user interactions are more
informative than others %(tagging is often more informative than
%commenting, video interactions are more informative than wall post
%interactions)
%we analyse trends in the relationship between the
%size of activity-based SAGs and informativeness
% (small groups can be
%highly informative while large groups are rarely informative).
and we show that activity informativeness varies drastically with type, size, 
and exposure.  In summary, this work demonstrates the previously untapped
predictive power of fine-grained social interaction and activity
features and the novel method of SAF to leverage them for
state-of-the-art social recommender systems.

