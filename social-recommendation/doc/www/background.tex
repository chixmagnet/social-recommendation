\emph{Collaborative filtering} (CF)~\cite{collab_filtering} is the
task of predicting whether, or how much, a user will like (or dislike)
an item by leveraging knowledge of that user's preferences \emph{as
well as those of other users}.  While collaborative filtering need not
take advantage of user or item features (if available), a separate 
approach of \emph{content-based filtering} (CBF)~\cite{newsweeder}
makes individual recommendations by generalizing from the
item features of those items the user has
explicitly liked or disliked.
% CBF approaches
%often use some form of learning and simply reduce to the machine
%learning tasks of classification (``will the user like a certain
%item?'') or regression (``how much will they like it?'').  
 What distinguishes
CBF from CF is that CBF requires item features to generalize whereas
CF requires multiple users to generalize; however,
CBF and CF are not mutually exclusive and recommendation systems often
combine the two approaches~\cite{fab}.   When a CF method makes use
of item and user features as well as multiple users, we refer to it as
CF although in some sense it may be viewed as a combined CF and CBF
approach.

We define \emph{social CF} (SCF) as the task of CF augmented
with additional social network information such as the following:
%that are available on social networking sites such as Facebook:
\begin{itemize}
\item Expressive personal profile content: gender, age, places lived, schools
attended; favorite books, movies, quotes; online photo albums (and associated comment text).
\item Explicit friendship or trust relationships.
\item Content that users have personally posted (often text, images, and links).
\item Content of public (and if available, private) interactions
between users (often text, images and links).
\item Evidence of external interactions between users such as being 
jointly tagged in photos or videos.
\item Expressed preferences (likes/dislikes of posts and links).
\item Group memberships (often for hobbies, activities, social or political discussion).
\end{itemize}
We note that CF is possible in a social setting without taking
advantage of the above social information, hence we include CF
baselines in our later experiments on SCF.  
%And this is an
%important reality check --- if SCF methods cannot beat standard CF
%methods, then clearly we have not been able to use additional social
%network content to our predictive advantage.

%Next we define common notation for the task of SCF followed by an
%in-depth discussion of prior work and baselines evaluated in this
%paper.

\subsection{Notation}

\label{sec:notation}

We present all algorithms for CF and SCF using the following
mathematical notation:
\begin{itemize}
\item $N$ users.  For methods that can exploit user features, we define 
an $I$-element user feature vector 
$\x \in \R^I$ (alternately if a second user is needed, $\z \in \R^I$).
For methods that do not use user feature vectors, we simply assume $\x$
is an index $\x \in \{ 1 \ldots N \}$ and that $I=N$.

\item $M$ items.  For methods that can exploit item features, we define
a $J$-element feature vector 
$\y \in \R^J$. The feature vectors for users 
and items can consist of any real-valued features as well as $\{0,1\}$
features like user and item IDs.
For methods that do not use item feature vectors, we simply assume $\y$
is an index $\y \in \{ 1 \ldots M \}$ and that $J=M$.
%$\x_{1 \ldots I}$ ($\z_{1 \ldots I}$).  
%$\y_{1 \ldots J}$.s

\item A (non-exhaustive) data set $D$ of single user \emph{preferences} of the form
$D = \{ (\x, \y) \to R_{\x,\y} \}$ where 
the binary \emph{response} 
$R_{\x,\y} \in \{ 0 \; \mbox{(dislike)}, 1 \; \mbox{(like)} \}$.

\item A (non-exhaustive) data set $C$ of \emph{co-preferences} (cases where
\emph{both} users $\x$ and $\z$ expressed a preference for $\y$ -- not
necessarily in agreement) derived from $D$ of the form
$C = \{ (\x, \z, \y) \to P_{\x, \z, \y} \}$ where co-preference class 
$P_{\x, \z, \y} \in \{ -1 \; \mbox{(disagree)}, 1 \;\mbox{(agree)} \}$.  
Intuitively, if \emph{both} user $\x$ and $\z$ liked or disliked item 
$\y$ then we say they \emph{agree}, otherwise if one liked the item and
the other disliked it, we say they \emph{disagree}.

\item A similarity rating $S_{\x,\z}$ between any users $\x$ and $\z$. This is used to summarize all social
interaction between user $\x$ and user $\z$ in the term $S_{\x,\z} \in
\R$.  A definition of $S_{\x,\z} \in \R$ that has been useful is the
following average-normalized measure of user interactions:
\begin{align}
\mathit{Int}_{\x,\z} & = \frac{\mbox{\# interactions between $\x$
and $\z$}}{\frac{1}{N^2} \sum_{\x',\z'} \mbox{\# interactions between $\x'$
and $\z'$}} \nonumber \\
S_{\x,\z} & = \ln \left( \mathit{Int}_{\x,\z} \right)
\end{align}
How ``\# interactions between $\x$ and $\z$'' is explicitly defined is
specific to a social network setting and hence we defer details of the
particular method user for evaluations in this paper to
Section~\ref{sec:interactions}.

We also define $S^+_{\x,\z}$, a \emph{non-negative} 
variant of $S_{\x,\z}$:
\begin{align}
S^+_{\x,\z} & = \ln \left( 1 + \mathit{Int}_{\x,\z} \right)
\end{align}
\item A set $\mathit{friends}_\x$ such that $\z \in \mathit{friends}_\x$
iff $\z$ is officially denoted as a \emph{friend} of $\x$ 
on the social network.
\end{itemize}

%% Joseph: what is this paragraph doing here???  Commented out.
%
%The matrix $R$ is a sparse $N \times M$ matrix of user ratings on
%items. The problem of recommendation is filling out the empty elements
%of this matrix, and this can be looked at as a linear regression
%problem. There are two general ways that this has been done
%previously, Content-based Filtering (CBF) and Collaborative Filtering
%(CF). Most traditional CBF methods learn in an explicit
%feature space, while most traditional CF methods learn in a latent
%feature space.

%% Joseph: this has now been covered in the Introduction, no need to
%%         include here.
%
%\subsection{Related work}
%
%There is a massive amount of related work on 
%SCF~\cite{matchbox,ste,lla,glfm,tf,sorec,sr,rrmf,bisim,socinf} embodying some of the
%ideas above, however there are a few aspects covered here, not covered
%in this related work:
%\begin{enumerate}
%\item Existing SCF methods \emph{cannot} capture some of the basic features that are used in standard CBF systems due to the inherent independent factorization between user and items (e.g., how much one user follows another) --- this is the motivation behind the \emph{hybrid} objectives.
%\item All methods \emph{except} for Matchbox~\cite{matchbox} ignore the issue of user and item features.  We extend the Matchbox approach above in our SCF methods. 
%\item \emph{None} of the methods that propose social regularization~\cite{ste,sr,rrmf,lla,glfm,socinf} incorporate user features into this regularization (as done above).
%\item Tensor-based factorizations such as~\cite{tf} use a full $K \times K \times K$ tensor for collaborative filtering w.r.t.\ tag prediction for users and items.  While our co-preference regularization models above were motivated by tensor approaches, we instead take an item-reweighted approach to the standard inner products to (a) avoid introducing yet more parameters and (b) as a way to introduce additional regularization in a way that supports the standard Matchbox~\cite{matchbox} CF model where prediction at run-time is made for a (user,item) pair, not for triples of (user,item,tag) as assumed in the tensor models.
%\end{enumerate}

Having now defined notation, we proceed to survey a number of 
CBF, CF, and SCF algorithms including all of those 
compared to or extended in this paper.

\subsection{Content-based Filtering (CBF)}

\label{sec:cbf}

Since our objective in this work is to classify whether a user likes
an item or not (i.e., a binary objective), we focus on binary
classification-based CBF approaches.  While a variety of
classifiers may work well, we choose the
\emph{support vector machine} (SVM)~\cite{svm} since it is
well-known for its state-of-the-art classification performance.

For the experiments in this paper, we use a \emph{linear} SVM
(implemented in the \emph{LibSVM}~\cite{libsvm} toolkit) with feature
vector $\f \in \mathbb{R}^F$ derived from $(\x,\y) \in D$, denoted as
$\f_{\x,\y}$.  A linear SVM learns a weight vector $\w \in
\R^F$ such that $\w^T \f_{\x,\y} > 0$ indicates a 
like (1) classification of $\f_{\x,\y}$ and $\w^T \f_{\x,\y} \leq 0$
indicates a dislike (0) classification.

A detailed list of features used in the SVM for the Facebook link
recommendation task evaluated in this paper are defined in
Section~\ref{sec:dataset} --- these include \emph{user features} such
as age and gender (binary) and \emph{item features} such as popularity
(number of times the item was shared).  Going one step beyond standard
CBF, our SVM features also include \emph{joint} user and item
features from the social network, in particular binary 
\emph{information diffusion}~\cite{inf_diffusion} features
for \emph{each} friend $\z \in \mathit{friends}_\x$ indicating if $\z$
liked (or disliked) $\y$.  Crucially we note that our SVM
implementation of CBF using social network features actually
represents a \emph{social CBF} extension since it can learn when a friend
$\z$'s preference for items are predictive of user $\x$'s preferences.

\subsection{Collaborative Filtering (CF)}

\subsubsection{$k$-Nearest Neighbor}
\label{sec:nn}

One of the most common forms of CF is the nearest neighbor
approach~\cite{bellkor}. 
%The $k$-nearest neighbor algorithm is a
%method of classification or regression that is based on finding the
%$k$-closest training data neighbors in the feature space nearest to a
%target point and combining the information from these neighbors ---
%perhaps in a weighted manner --- to determine the classification or
%regression value for the target point.
There are two main variants of nearest neighbors for
CF, \emph{user-based} and \emph{item-based} --- both
methods generally assume that no user or item features are provided,
so here $\x$ and $\y$ are simply respective user and item indices.
When the
number of users is far fewer than the number of items, it has been
found that the user-based approach usually provides better predictions
as well as being more efficient in computations~\cite{bellkor};
this holds for the evaluation in this paper, so 
we focus on the user-based approach.

Given a user $\x$ and an item $\y$, % let $N(\x:\y)$ 
% ebonilla: notation clashed before 
let ${\mathcal N}(\x:\y)$
be the set of
\emph{user} nearest neighbors of $\x$ that have also given a rating
for $\y$
%$N(\y:\x)$ be the set of
%\emph{item} nearest neighbors of $\y$ that have also been rated by
%$\x$ 
and let $S_{\x,\z}$ be the cosine similarity (i.e., normalized
dot product) between two vectors of ratings for users $\x$ and $\z$
(when both have rated the same item).  
%measure of similarity rating between users $\x$ and $\z$
%defined prevui
%$S_{\y,\y'}$ be the cosine similarity (i.e., normalized
%dot product) between two vectors of item ratings for $\y$ and $\y'$ with
%an index for each user that has rated both items.  
Following~\cite{bellkor},
the predicted rating $\hat{R}_{\x,\y} \in [0,1]$ that the user $\x$
gives item $\y$ can then be calculated as
%\begin{itemize}
%\item {\bf Item-based similarity nearest neighbor:}
\begin{equation}
%\hat{R}_{\x,\y} = \frac{\sum_{\y' \in N(\y:\x)} {S_{\y,\y'} R_{\x,\y'}} } {\sum_{\y' \in N(\y:\x)}{S_{\y,\y'}}} .
\hat{R}_{\x,\y} = \frac{\sum_{\z \in N(\x:\y)} {S_{\x,\z} R_{\z,\y}} } {\sum_{\z \in N(\x:\y)}{S_{\x,\z}}}\end{equation}
%\end{itemize}

%\begin{comment}
%This applies to the MovieLens 1 Million dataset as well. For the
%MovieLens 100,000 dataset, the number of items is larger than the
%number of users, and the user-based approach has been found to perform
%better.
%\end{comment}

\subsubsection{Matrix Factorization (MF) Models}
\label{sec:mf}

Another common approach to CF attempts to factorize an (incomplete)
matrix $R$ of dimension $I \times J$ containing observed ratings $R_{\x,\y}$ 
(note that $\x$ and $\y$ are assumed to row and column indices)
into a product $R \approx U^T V$ of real-valued
rank-$K$ matrices $U$ and $V$:
\begin{equation*}
U = 
\begin{bmatrix}
  U_{1,1} & \hdots  & U_{1,I} \\
  \vdots  & U_{k,i} & \vdots  \\
  U_{K,1} & \hdots  & U_{K,I} \\
\end{bmatrix}
\qquad 
V = 
\begin{bmatrix}
  V_{1,1} & \hdots  & V_{1,J} \\
  \vdots  & V_{k,j} & \vdots  \\
  V_{K,1} & \hdots  & V_{K,J} \\
\end{bmatrix}
\end{equation*}
In this initial MF setting, we \emph{do not} leverage user and item features;
hence, the $\x$ and $\y$ indices simply pick out the respective
rows and columns of $U$ and $V$ such that $U_\x^T V_\y$ acts as a
measure of affinity between user $\x$ and item $\y$ in their respective % ebonilla
$K$-dimensional latent spaces $U_\x$ and $V_\y$.
%Then we see that the
%basic idea behind matrix factorization techniques is to project the
%user $\x$ and item $\y$ into some $K$-dimensional space where the dot
%product in this space indicates the relative affinity of $\x$ for
%$\y$.  
%Because this latent space is low-dimensional, i.e., $K \ll
%I$ and $K \ll J$, similar users and similar items will tend to be
%projected ``nearby'' in this $K$-dimensional space.

However, there remains the question of how we can learn $U$ and $V$
given that $R$ is incomplete (i.e., it contains missing entries since
$D$ is generally non-exhaustive).  The answer is simple: we need only
define a reconstruction error objective we wish to minimize as a
function of $U$ and $V$ and then use gradient descent to optimize it;
formally then, we can optimize the following MF objective~\cite{pmf}:
\begin{align}
\sum_{(\x,\y) \in D} \frac{1}{2} (R_{\x,\y} - U_\x^T V_\y)^2 \label{eq:basic_mf}
\end{align}
While this objective is technically bilinear, 
we can easily apply an
\emph{alternating gradient descent} approach to approximately
optimize it and determine good projections $U$ and
$V$ that (locally) minimize the reconstruction error of the observed
responses $R_{\x,\y}$ (see e.g.~\cite{pmf}). % ebonilla

%These are well-known MF approaches to CF, however in the context of
%social networks, we'll need to adapt them to this richer setting to
%obtain SCF approaches.  We discuss these extensions next.

\subsubsection{Social Collaborative Filtering (SCF)}
\label{sec:scf_original}
%These are MF methods that use the social network and do recommendation,
%note that the GLFM and Bidirectional similarity papers do not meet these
%requirements

In this work on \emph{social CF} (SCF), we % opt to % ebonilla 
focus on extending
MF-based SCF 
% objectives since 
% gradient descent MF optimization
%approaches allow us a flexible language to design and optimize
%objective functions that take into account a vast array of social
%network information.
% ebonilla
approaches as  they allow us to incorporate flexible 
objective functions that take into account a vast array of social
network information. Additionally, we can learn the parameters 
of the proposed algorithms from data by using
gradient-based optimization.


To date, there are essentially two general classes of MF methods
applied to SCF of which the authors are aware.  
%All of the social MF
%methods defined to date
%\emph{do not} make use of user or item features and hence $\x$ and
%$\z$ below should be treated as user indices as defined previously for
%the non-feature case.
The first class of social MF methods can be termed as \emph{social
regularization} approaches in that they % somehow  % ebonilla
constrain the latent
projection of users according to social network information.  There
are two closely related social regularization methods that directly
constrain $U_\x$ and $U_\z$ for user $\x$ and $\z$ based on evidence
$S_{\x,\z}$ of interaction between $\x$ and $\z$.  The first class of
methods are simply termed \emph{social
regularization}~\cite{lla,socinf} where $\la \cdot, \cdot \ra$ denotes an inner
product:
% Note that these previous methods do not use user and item features
%\begin{itemize}
%\item {\bf Social regularization~\cite{lla,socinf}}:
\begin{align}
\sum_{\x} & \sum_{\z \in \mathit{friends}_\x} \frac{1}{2} (S_{\x,\z} - \la U_\x, U_\z \ra)^2 \label{eq:simple_sr}
%\text{,} % ebonilla
\end{align}
% ebonilla
%where  $\mathit{friends}_\x$ refers to the set of friends of user $\x$, or more generally, the 
%set of users to which $\x$ is connected directly in the social network.
%

The second class of methods are termed 
\emph{social spectral regularization}~\cite{sr,rrmf}:
%\item {\bf Social spectral regularization~\cite{sr,rrmf}}:
\begin{align}
\sum_{\x} & \sum_{\z \in \mathit{friends}_\x} \frac{1}{2} S^+_{\x,\z} \| U_\x - U_\z \|_2^2 \label{eq:simple_spectral_sr}
\text{.}
\end{align}
%\end{itemize}
We refer to the latter as \emph{spectral} regularization methods since they are
identical to the objectives used in spectral clustering~\cite{spectral}.
The idea behind both variants of social regularization should be apparent:
the larger $S_{\x,\z}$ or $S^+_{\x,\z}$, the more $U_\x$ and $U_\z$ need to
be similar (according to slightly different metrics) 
in order to minimize the given objective.

The {\it SoRec} system~\cite{sorec} proposes a slight twist on social
spectral regularization in that it learns a third $N \times N$ (n.b., $I = N$)
\emph{interactions matrix} $Z$, and uses $U_\z^T Z_\z$ to predict user-user
interaction preferences in the same way that standard CF uses $V$ in
$U_\x^T V_\y$ to predict user-item ratings.  {\it SoRec} also uses a
sigmoidal transform $\sigma(o) = \frac{1}{1 + e^{-o}}$ since $\bar{S}_{\x,\z}$
is $S_{\x,\z}$ restricted to the range $[0,1]$ 
(e.g., $\bar{S}_{\x,\z} = \sigma(S_{\x,\z})$):
%
%\begin{itemize}
%\item {\bf SoRec regularization~\cite{sorec}}:
%
\begin{align}
\sum_{\z} & \sum_{\z \in \mathit{friends}_\x} \frac{1}{2} (\bar{S}_{\x,\z} - \sigma(\la U_\x, Z_\z \ra))^2 
\end{align}
%\end{itemize}

The second class of SCF MF approaches represented by the single
exemplar of the {\it Social Trust Ensemble} (STE)~\cite{ste} can be termed as a
\emph{weighted friend average} approach since this approach simply composes a
prediction for item $\y$ from an $\alpha$-weighted average ($\alpha
\in [0,1]$) of a user $\x$'s predictions \emph{as well as} their friends
($\z$) predictions (as evidenced by the additional $\sum_\z$ in the
objective below):
%\begin{itemize}
%\item {\bf Social Trust Ensemble (STE)~\cite{ste}}:
\begin{align}
\sum_{(\x,\y) \in D} \frac{1}{2} (R_{\x,\y} - \sigma (\alpha U_\x^T V_\y + (1 - \alpha) \sum_{\z \in \mathit{friends}_\x} U_\x^T V_\z))^2 
\end{align}
%\end{itemize}
As for the MF CF methods, all MF SCF methods can be optimized by alternating
gradient descent on the respective matrix parameterizations; we refer
the interested reader to each paper for further details.

