
Interactions between users and their friends reveal many behaviors unique to virtual online social interactions. Finding commonalities in these interactions have revealed many common user behaviors.


With the average Facebook user having 130 friends\footnote{https://www.facebook.com/press/info.php?statistics}, users naturally characterise and divide their interaction and attention amongst their friends. Two studies of over 10 million Facebook users shows how users interact and divide their attention amongst small subsets of Facebook friends.~\cite{wilson2009user} built interaction graphs that better reflect real-world interactions. These graphs show more favorable network properties to the full Facebook friendship graphs.~\cite{backstrom2011center} studied two types of attention: communication interaction and viewing attention (e.g. looking at profiles or photos). Users' communication attention is focused on small number of friends, but viewing attention is dispersed across all friends. Demographics also affect attention, such as male users are more focused in their attention than females. 


Which friends a user interacts with on social networking sites depend on some defining factors.~\cite{gilbert2009predicting} asked 35 Facebook users to assess their friendship strengths by answering questions on trust of friends. Friendship strength seems to be (most to least) predictable on seven dimensions: intimacy, intensity, duration, social distance, services, emotional support, and structural information. These friendship strengths are also partially observed by~\cite{singla2008yes} on MSN chat log and web search log. MSN users who chat with each other regularly are more likely to share interests and other personal characteristics, and have stronger relationships. Real-world interactions further strengthens friendship on Facebook as found by~\cite{brandtzag2011facebook}. Furthermore, Facebook interactions in general does not displace offline interactions and facilitates offline family interactions.~\cite{yang2011culture} observed cultural differences define the type of interactions users engage in. Cultural variables seem to create more variance in interactions on social networking sites than demographic variables such as age or gender. One observed difference is users in Asian countries preferred to post professional networking and question status updates compared to users in Western countries.~\cite{tufekci2010who} studied how online interactions on social networking sites leads to creating new friendships and maintaining past friendships. An interesting insight is how race affects formation of new friendships and interactions in an otherwise diverse population.


The spread of information on social networks can be determined by influence and interaction.~\cite{ver2011stops} investigated information diffusion in the Digg social network using epidemiology models of the spread of contagious diseases. The `social infection’ process is contrary to epidemiological models: the likelihood that a user votes on a story reduces with repeated exposure.\cite{lerman2010information} observed the effects of social dynamics on information flow of Digg and Twitter social networks. These two networks share similarities with friendship networks and spread of news information within these networks. However, Digg’s dense friends network allows stories to spread quickly and to unconnected users on Digg’s front page. Twitter’s network is less dense, but stories persist longer and penetrates deeper into the network. For Twitter,~\cite{asur2011trends} observed for trends of topics to persist, large active groups of users are needed to contribute and retweet. The greater the propagation of topics through Twitter, thereby accumulating more users and retweets, the longer a topic is trending. Influential users are crucial to trending topics by their active contribution and retweet-ratio.


Influence is an important user trait that results from interactions.~\cite{cha2010measuring} compares three measures of influence of Twitter users: indegree (followers of a user), retweet (retweets containing user’s name) and mention (mentions of a user). Retweet and mention influences reveal the most influential users span a variety of topics and time. Top Twitter users are highly influential and devote high personal involvement; also observed by~\cite{asur2011trends}.~\cite{li2011casino} presented an algorithm to analyse influences in social networks based on positive and negative interactions. Their algorithm was successful to identify both influential as well as conforming individuals in several different types of social networks.~\cite{saez2011high} analysed three social networks: Enron email data set, Facebook and Twitter. Facebook and Twitter shows high correlation between incoming and outgoing activities. To become an influential user on Facebook or Twitter, a user needs to generate outgoing activity.~\cite{sun2011participation} formulated social influence of forum users of TripAdvisor into a maximalization problem. One algorithm of this formulation uses the social influence of posters to further disseminate data by increasing the likelihood of future participation of other users in threads.~\cite{cosley2010sequential} modeled social influence and its spread in Wikipedia contributors using a probabilistic framework. The study compared the effects of sampling procedures resulting in ordinal data and snapshots. Algorithms simulating ordinal time from snapshots were explored to improve accuracy in approximations.


User interactions can be predicted using data mining and machine learning techniques.~\cite{gomes2011social} represented over 1,000 Facebook users' interactions as rules. These handcrafted rules followed a formal symbolic language developed by a behavioral psychologist. From data mining these rules, ranking them on confidence and support, the study found users are more likely to `like' another user's post or comment than to actively comment on it, and that the action of a `like' or `comment' by one user does not affect the involvement of other users in social interactions.~\cite{nori2011exploiting} examines the predictability of user actions on twitter from actions in both twitter and del.icio.us. The study proposes an bipartite graph model and uses its Laplacian to improve predictions over LDA and tensor models.




%Facebook profiles also determines perceptions of real sociability and attractiveness.~\cite{tong2008too} recruited 153 Facebook users to evaluate their perception of crafted Facebook profiles. The number of friends a user has influences perception of social attractiveness, physical attractiveness and extraversion.~\cite{utz2010showme} observed further cues of influencing impressions of profiles on Hyves, a Dutch social networking site comparable to Facebook: information generated by the user, the user’s friends, and the system. The combination of these cues have strong effects on perceptions of popularity, introversion, extroversion, communal orientation, similarity of friends, and other traits.


\textbf{This paper (Wilson et al.~\cite{wilson2009user}) contains an excellent description of the Facebook user interactions - might be worth considering to add something like this to our paper:}


\textbf{Each profile includes a message board called the “Wall” that serves as the primary asynchronous messaging mechanism between friends. Users can upload photos, which must be grouped into albums, and can mark or “tag” their friends in them. Comments can also be left on photos. All Wall posts and photo comments are labeled with the name of the user who performed the action and the date/time of submission. Another useful feature is the Mini-Feed, a detailed log of each user’s actions on Facebook over time. It allows each user’s friends to see at a glance what he or she has been doing on Facebook, including activity in applications and interactions with common friends. Other events include new Wall posts, photo uploads and comments profile updates, and status changes. The Mini-Feed is ordered by date, and only displays the 100 most recent actions.}
