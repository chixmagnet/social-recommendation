% Paper summaries for related works section of ICDSM-2012 paper 
% PC, Jan 2012

% Algorithm to analyse influences and conformity based on both
% positive and negative interactions - not sure if relevant to our
% work.

@inproceedings{Li:2011:CTC:2063576.2063721,
 author = {Li, Hui and Bhowmick, Sourav S. and Sun, Aixin},
 title = {CASINO: towards conformity-aware social influence analysis in online social networks},
 booktitle = {Proceedings of the 20th ACM international conference on Information and knowledge management},
 series = {CIKM '11},
 year = {2011},
 isbn = {978-1-4503-0717-8},
 location = {Glasgow, Scotland, UK},
 pages = {1007--1012},
 numpages = {6},
 url = {http://doi.acm.org/10.1145/2063576.2063721},
 doi = {http://doi.acm.org/10.1145/2063576.2063721},
 acmid = {2063721},
 publisher = {ACM},
 address = {New York, NY, USA},
 keywords = {conformity, influence, signed edge, social networks, topic-based subgraphs, twitter},
} 

Li et al.~\cite{Li:2011:CTC:2063576.2063721} recently presented an
algorithm to analyse influences in social networks. Using both positive
interactions (such as agreements and trust) as well as negative
interactions (like distrust and disagreements) between individuals in
a network, they calculate an influence and a conformity index for each
individual. Their CASINO algorithm was successful to identify both
influential as well as conforming individuals in several
different types of social networks.

% -------------------------------

% Analysis of incoming and outgoing activities, similar to what we do,
% on Facebook, Twiter and Enron email.

@inproceedings{saez2011high,
  title={High Correlation between Incoming and Outgoing Activity: A Distinctive Property of Online Social Networks?},
  author={Saez-Trumper, D. and Nettleton, D. and Baeza-Yates, R.},
  booktitle={Fifth International AAAI Conference on Weblogs and Social Media},
  year={2011}
}

Saez-Trumper and Nettleton~\cite{saez2011high} analysed three
different types of social networks with regard to the incoming and
outgoing activity of users, and they found that in both data from
Facebook and Twitter there is a high correlation between incoming and
outgoing activities. However, only a low correlation on these
activities was found in the Enron email data set. In the Facebook
data, the outgoing activity was calculated according to the number of
posts done by a user, while the incoming activity was the number of
posts received. For the Twitter data, outgoing activity was the number
of tweets done while the incoming activity was the number of followers
a user had on Twitter. With the Enron data sets, outgoing and incoming
activities were calculated as the number of emails sent and received.
The conclusion of this analysis was that in social network media one
important factor that can influence if one becomes an influential user
is to generate outgoing activity, i.e.\ to for example post on other
user's wall.

% ----------------------

% Predicting tie strength, using several dimensions, based on Facebook
% data and follow-up interviews

@inproceedings{gilbert2009predicting,
  title={Predicting tie strength with social media},
  author={Gilbert, E. and Karahalios, K.},
  booktitle={Proceedings of the 27th international conference on Human factors in computing systems},
  pages={211--220},
  year={2009},
  organization={ACM}
}

Gilbert and Karahalios~\cite{gilbert2009predicting} analysed a data
set consisting of over 2,000 Facebook friendships with the aim to
predict the strengths of ties in social media data. This data set was
manually collected from the Facebook accounts of 35 users who
additionally answered five questions to assess tie strengths. Included
in these five questions were, for example, `How strong is your
friendship with this person' and `How would you feel asking this person
to load you \$100 or more?'. Based on the data collected, 74 variables
were used in a linear model to predict tie strength. Extending earlier
work on tie strength, conducted mostly in the social sciences, Gilbert
and Karahalios defined seven dimensions for tie strengths. They found
that the three most predictive dimensions were intimacy (32.8\%,
measured as the recency of the last communication, number of common
friends and intimate words used), followed by intensity (19.7\%,
measured by the number of words posted on a friends Facebook wall, the
number of outbound posts and the depth of communication threads
between friends), and duration (16.5\% measured as the time when the
first communication between friends occurred). The four less predictive
dimensions were social distance, services, emotional support and
structural information such interest overlaps or common group
memberships~\cite{gilbert2009predicting}. Classifying tie strength
into weak and strong as has previously been proposed by social science
researchers, the model proposed by Gilbert and Karahalios achieved an
accuracy of over 85\% when distinguishing between weak and strong ties.

% ----------------------

% modelling of spread of information through networks based on
% epidemiological models.

@inproceedings{ver2011stops,
  title={What Stops Social Epidemics?},
  author={Ver Steeg, G. and Ghosh, R. and Lerman, K.},
  booktitle={Proc. 5th Int. Conf. on Weblogs and Social Media},
  year={2011}
}

Ver Steeg et al.~\cite{ver2011stops} recently investigated how
information is spread though social networks by employing models that
originally have been developed in the area of epidemiology to model
the spread of contagious diseases. They extracted network data from
the social news site Digg constraining around 3,500 stories and nearly
140,000 users who actively voted for at least one of these stories.
They then built a friendship network with more than 1.7 million
directed links between users according who they nominated as their
friends. Based on this network they then analysed how stories spread
as users vote for a story which is then exposed to their friends in
Digg. The spread of a story through Digg's network is described as a
`social infection' process, and because several user can initially
vote for a  story independently from others, many cascades are started
for the same story. Interestingly, contrary to epidemiological models,
the likelihood that a user votes for a story (i.e.\ recommends a
story) reduces with repeated exposure (i.e.\ the more of their friends
vote for the story). As a result, in their analysis, Ver Steeg et
al.\ discovered that most stories do no spread through more then 0.1\%
of Digg's network structure. A theoretical analysis comparing the real
Digg data with simulated data confirms these empirical results.

% ----------------------

% Analysis of different interactions on Facebook: communication versus
% viewing (similar to our real / virtual friends?)

@inproceedings{backstrom2011center,
  title={Center of attention: How facebook users allocate attention across friends},
  author={Backstrom, L. and Bakshy, E. and Kleinberg, J. and Lento, T.M. and Rosenn, I.},
  booktitle={Proc. 5th International Conference on Weblogs and Social Media},
  year={2011}
}

Backstrom et al.~\cite{backstrom2011center} propose a measure to
express how users divide their attention across their friends in
Facebook. They classify attention into two groups: communication,
which includes sending messages, leaving comments on a photo or video,
and wall posts; and viewing of profiles and photos of another user.
Using a Facebook data set covering the full year of 2010, they find
that communication interactions are much more focussed with a
higher portion of an individual's attention going towards a small
number of friends compared to viewing attention which is much more
dispersed across all friends of a user. Drilling down into the data,
they find that older user are having a more focussed viewing behaviour
while at the same time exhibiting a more disperse communication
behaviour. Similarly, male users are more focussed in their
attention than females. Further differences were found in the
interactions between genders, and between individuals that had
different relationship status.

% ----------------------

% represent different social interactions using rules in the Mechner
% language developed by a behavioural psychologist, and evaluated these
% rules on Facebook data. Rules include post on wall, like a post,
% comment on a post, etc.

@inproceedings{gomes2011social,
  title={Social interactions representation as users behavioral contingencies and evaluation in social networks},
  author={Gomes, A.K. and da Graca C Pimentel, M.},
  booktitle={Semantic Computing (ICSC), 2011 Fifth IEEE International Conference on},
  pages={275--278},
  year={2011},
  organization={IEEE}
}

Gomes and Pimental~\cite{gomes2011social} recently presented an
approach to represent social interactions using a set of hand-crafted
rules. These rules are written in a formal symbolic language, called
the Mechner language as developed by a behavioural psychologist. Using
data collected from over 1,000 Facebook users, in total over 300,000
interactions, the authors evaluated a set of rules using the measures
support, confidence and cosine correlation, as used in data mining.
Ranking these rules according to confidence and support they found
that users are more likely to `like' another user's post or comment
than to actively comment on it, and that the action of a `like' or
`comment' by one user does no affect the involvement of other users in
social interactions.

% ----------------------

% Interesting paper to read!

% Analyses a large-scale Facebook data set with regard to user
% interactions and defines an interaction graph which is then compared
% to the full friends network.

@inproceedings{wilson2009user,
  title={User interactions in social networks and their implications},
  author={Wilson, C. and Boe, B. and Sala, A. and Puttaswamy, K.P.N. and Zhao, B.Y.},
  booktitle={Proceedings of the 4th ACM European conference on Computer systems},
  pages={205--218},
  year={2009},
  organization={Acm}
}

Wilson et al.~\cite{wilson2009user} investigate the question if social
links can be used as valid indicators of real-world social
interactions. They evaluate detailed user interactions using Facebook
data from more than 10 million users containing over 940 million
social links and 24 million interactions. They found that user
interactions on Facebook are significantly different from real social
links between users. Many users tend to interact with a small sub-set
of their Facebook friends, while not having any interactions with more
than half of their other friends. They then propose interaction graphs
which quantify user interactions on a social network by only
connecting two users if they have been actively interacting with each
other, parameterised by a minimum number of $n$ interactions in a time
period $t$. They compare these interaction graphs with the full
network graph which connects all users who are friends on Facebook
with each other. The presented results indicate significant
differences in network properties, such a larger network diameters,
lower clustering coefficients, and higher assortativity in interaction
graphs compared to full friendship graphs.


% This paper contains an excellent description of the Facebook
% user interactions - might be worth considering to add something
% like this to our paper:
%
% 
%Each profile includes a message board called the �Wall�
%that serves as the primary asynchronous messaging mechanism
%between friends. Users can upload photos, which must
%be grouped into albums, and can mark or �tag� their friends
%in them. Comments can also be left on photos. All Wall posts
%and photo comments are labeled with the name of the user
%who performed the action and the date/time of submission.
%Another useful feature is the Mini-Feed, a detailed log of
%each user�s actions on Facebook over time. It allows each
%user�s friends to see at a glance what he or she has been
%doing on Facebook, including activity in applications and
%interactions with common friends. Other events include new
%Wall posts, photo uploads and comments profile updates, and
%status changes. The Mini-Feed is ordered by date, and only
%displays the 100 most recent actions.
