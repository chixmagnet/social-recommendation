\label{sec:methodology}

Having now described the data we have at our disposal, we proceed
to describe the evaluation methodology.

% Need to previously define G and U

Our primary objective in this paper is to evaluate what user-to-user
interactions and shared user traits (interests, history, demographics)
are predictive of common preferences among friends.  For the purpose
of this paper, we simply define an \emph{expressed unary preference}
as the click of the \emph{like button} on a link, post, photo, or
video in Facebook, with each set of liked items respectively defined as
$L_\link$, $L_\post$, $L_\photo$, $L_\video$, and all likes as $L_\all
= L_\link \cup L_\post \cup L_\photo \cup L_\video$.

Given this candidate set of preferences and a set of Facebook users
$U$ (where the subset of LinkR app users for whom 
we have complete interaction data with all 39,000+ friends
is denoted by $U_\app \subseteq U$), the main
question we want to answer is the following: 

\begin{quote}
For a random user $u \in U_\app$, a set of candidate items to like $L
\subseteq L_\all$, a subgroup $G \subseteq U$ of user's friends
meeting some \emph{group criteria} (e.g., a group of at most $n$
members) and the subset of items $L_{G,C} \subseteq L$ liked by $G$
meeting some \emph{link critera} $C$ (e.g., links liked at least $k$
times by members of $G$), what is the probability that $u$ will like
any item $i \in L_G$?
\end{quote}

Defining, $L_u \subseteq L$ as the subset of candidate items liked
by user $u$, this query can be computed very easily as follows
\begin{equation}
P(u \mbox{ likes } i | i \in L_{G,C}) = \frac{|L_u \cap L_{G,C}|}{|L_{G,C}|}
\end{equation}
where $|\cdot|$ is the usual set cardinality operator.

There are a wide range of critera we can use to define this
query which we enumerate now:
\begin{itemize}
\item \textbf{Like Type:} In the query, we can define the like
set $L$ as any one of $L_\link$, $L_\post$, $L_\photo$, $L_\video$,
or $L_\all$.
\item \textbf{Group Criteria:}  We can define 
    groups $G$ according to various relationships between users where
    any user $v$ in $u$'s friend group $G$ must meet all specified
    critera:
  \begin{itemize}
    \item \textbf{Interactions:} We can define 
    groups $G$ according to social interactions between users:
      \begin{itemize}
      \item \textbf{Interaction Modality:} User $u$ can interact with
       user $v$ via \textit{links}, \textit{posts}, 
       \textit{photos}, and \textit{videos}.
      \item \textbf{Interaction Action:} A user $u$ can
      \textit{comment} on or \textit{like} user $v$'s item,
      or \textit{tag} user $v$ on an item 
      (with the one exception in Facebook that $u$ cannot tag
      a link with user $v$ -- for the obvious reason that this is
      nonsensical).
      \item \textbf{Interaction Directionality:} Motivated
      by~\cite{saez2011high}, we can look
      at \textit{incoming} and \textit{outgoing} interactions, i.e.,
      if user $u$ comments on, tags, or likes user $v$'s item,
      then $u$ is in the set of incoming interactions for $v$
      and $v$ is in the set of outgoing interactions for $u$.
      \item \textbf{Real vs. Virtual:} \cite{brandtzag2011facebook}
      defines real vs. virtual interactions as those interactions
      occurring between friends who spend any time together offline
      (real) and those interactions with friends that are only online
      (virtual).  We define the \textit{real} friend group of $u$ as
      those friends $v$ who are tagged in the same photo or
      video, and all other friends in the \textit{virtual} friend group
      of $u$.
    \end{itemize}
    \item \textbf{Interests:} We can define groups $G$ according to
      those having common interests with $u$; in this paper there was
      sufficient data to evaluate five interest types: membership in
      Facebook \textit{groups} (e.g., fan clubs), Facebook
      \textit{page likes} (e.g., the ``re-elect Obama in 2012'' page),
      and usual interests such as \textit{movies}, \textit{music}, and
      \textit{television}.
    \item \textbf{History:} We can define groups $G$ according to
      those having common historical traits with $u$; in this paper
      there was sufficient data to evaluate two history groups:
      people who've attended the same \textit{schools} for education,
      and people who have had the same \textit{employer}.
    \item \textbf{Group Size:} We can restrict our group definition
      to look at small vs. large groups by restricting the group
      size for $u$ to be a maximum of $n$ friends.  We hypothesize 
      that small groups may be more predictive than large groups since
      they define narrower interest -- we later revisit this hypothesis
      in light of our forthcoming analysis.
      % in Section~\ref{sec:interest_history}.
    \item \textbf{Demographics:} We can restrict our group definition
      according to usual demographic criteria such as \emph{gender}.
      The user base was sufficiently narrow in this paper (most were
      University students) such that analysis by \emph{age} and other
      traits did not yield sufficient differentiation for
      interpretable evaluation.%; hence only demographic analysis by
      %gender is presented in Section~\ref{sec:demographics}.
  \end{itemize}
\item \textbf{Link Criteria:} Our main link criteria restriction $C$
      used to form the group liked item set $L_{G,C}$ from item set
      $L$ and user group $G$ was to only include links that had
      \emph{at least $k$ likes} among the members of $G$.  
      %One might
      %hypothesize that links liked by more people in $G$ (i.e., higher
      %$k$) would directly correlate with user $u$'s preferences -- we
      %revisit this hypothesis in light of our forthcoming analysis.
      \cite{Romero2011hashtag} refer to plots vs. $k$ as
      \emph{exposure} curves since they provide a rough indicator of
      $u$'s exposure to a liked item.
      %Section~\ref{sec:interactions}.
\end{itemize}

Regarding statistical aspects of the analysis, 
because we compute $P(u \mbox{ likes } i | i \in L_{G,C})$ for each
user $u \in U_\app$, we report the mean over $u$ and also compute 95\%
confidence intervals on this average probability.  Cases where
$L_{G,C} = \emptyset$ were discarded as these lead undefined
probabilities; in cases where fewer than 10 users $u$ had well-defined
probabilities, we omit these results altogether.  Confidence intervals
are omitted on some graphs where none of the results were
statistically significant w.r.t.\ these intervals (i.e., all user
trait results) --- in these cases,
we simply analyze the graphs to identify general trends across
multiple data points.

With the data and methodology now defined including all of the
dimensions for our analysis, 
%To better understand these subleties and to understand what
%what social interactions and user traits reflect common
%preferences on Facebook, 
we now proceed to an in-depth evaluation of various
queries such as those outline in Section~\ref{sec:introduction}.

