\documentclass[letterpaper]{article}
\usepackage{aaai}
\usepackage{comment}
\usepackage{times}
\usepackage{helvet}
\usepackage{courier}
\usepackage{amsmath,amsfonts,amssymb,amsthm}
\usepackage{array}
\usepackage{amsmath,amssymb}
\usepackage{epsfig,subfigure}
%\usepackage{hyperref}
\frenchspacing

\newcommand{\link}{\mathit{link}}
\newcommand{\post}{\mathit{post}}
\newcommand{\photo}{\mathit{photo}}
\newcommand{\video}{\mathit{video}}
\newcommand{\all}{\mathit{all}}
\newcommand{\app}{\mathit{app}}

% place holder spacing hacks
\newcommand{\secmoveup}{\vspace{-1.2mm}}                %{\vspace{-0.12in}}
\newcommand{\bigsecmoveup}{\secmoveup\vspace{-.0mm}}   %{\vspace{-0.08in}}
\newcommand{\textmoveup}{\vspace{-0mm}}               %{\vspace{-0.08in}}
\newcommand{\bigtextmoveup}{\textmoveup\vspace{-0.0in}} %{\vspace{-0.06in}}
\newcommand{\itemmoveup}{\vspace{-0mm}}              %{\vspace{-0.04in}}
\newcommand{\eqmoveup}{\vspace{-0.0in}}                 %{\vspace{-0.16in}}
\newcommand{\captionmoveup}{\eqmoveup\vspace{-0.0in}}   %{\vspace{-0.16in}}
\newcommand{\refitemmoveup}{\vspace{-0mm}}            %{\vspace{-0.16in}}
% hold but hide chunks of text
\newcommand{\eat}[1]{}

\newcommand{\true}{\mathit{true}}
\newcommand{\false}{\mathit{false}}
\def\argmax{\operatornamewithlimits{arg\,max}}
\def\argmin{\operatornamewithlimits{arg\,min}}

\setcounter{secnumdepth}{0}

\begin{document}
% The file aaai.sty is the style file for AAAI Press 
% proceedings, working notes, and technical reports.
%
\title{Which Social Interactions and User Traits Reflect \\Common Preferences on Facebook?}
\author{Anonymous}
%\author{Scott Sanner\\
%NICTA \& the ANU\\
%Canberra, Australia\\
%{\tt ssanner@nicta.com.au}
%\And
%Khoi-Nguyen Tran\and
%Lexing Xie\and
%Peter Christen\\
%Research School of Computer Science\\
%Australian National University\\

%{\tt \{khoi-nguyen.tran,lexing.xie,peter.christen\}@anu.edu.au}
%}
\maketitle
\begin{abstract}
%\begin{quote}
Social networks such as Facebook provide access to a 
rich set of user
preferences (likes of links, posts, photos, videos) and user traits
and interactions (conversation streams, tagging, group memberships,
interests, personal history and demographic data). However, which 
interactions or common traits among friends are actually reflective of
their common preferences, explicitly expressed by their likes?  In this work, we
evaluate these questions using a data set of over 100 Facebook users
and their complete interactions with over 39,000+ 
friends over a four month study period. Our analysis shows a large
variation in the way different user interactions and traits are
reflected in common interests and based on this we conclude with a
discussion of the underlying social phenomena that appear to
contribute to these variations.
%\end{quote}
\end{abstract}

\section{Introduction}

\input introduction

\section{Data Description}

\input data-description

\section{Methodology}

\input methodology

\section{Evaluation}

\input evaluation

\section{Related Work}

\input related-work

\section{Conclusions}

\input conclusions

%\section{Acknowledgements}

\bibliography{icwsm11_relpred}
\bibliographystyle{aaai}
\end{document}