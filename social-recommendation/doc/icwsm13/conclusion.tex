%!TEX root = document.tex

We proposed Social Affinity Filtering (SAF) as a new method for social
recommendation.  We first defined social affinity groups (SAGs) of a
target user by analyzing their fine-grained interactions and
activities.  Then we learn which SAGs are most predictive of the
target user's preferences leading to SAF.  We evaluated the proposed
algorithm on a dataset collected from a Facebook App we built, showing
that SAF yields 6\% absolute improvement in accuracy with respect to a
state-of-the-art social recommendation engine.  Furthermore, we
quantified the relative importance of interactions and activities for
recommendation as well as analysing what properties made some SAGs
more predictive than others.  Among many insights, our results show
that video and photo interactions are more predictive than other
modalities, and outgoing interactions are more predictive than
incoming, and that smaller social groups are more predictive than
larger ones.  Future directions of research can investigate the nature
of interactions by measuring the level of user engagement -- e.g. if
videos are inherently more engaging, or examine the nature of social
groups via additional metrics -- e.g. the social network within
members of the group, or activity level of the group.
