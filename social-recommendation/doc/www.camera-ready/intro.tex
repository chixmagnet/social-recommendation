Given the vast amount of content available on the Internet, finding
information of personal interest (news, blogs, videos, movies, books,
etc.) is often like finding a needle in a haystack.  Recommender
systems based on \emph{collaborative filtering}
(CF)~\cite{collab_filtering} aim to address this problem by 
leveraging the preferences of similar users in a user population.

%by 
%automatically identifying similar users and items in order to leverage
%the collective preferences of a user population.

%and leveraging the preferences of similar users in a user population.

%a user population under the assumption that similar
%users will have similar preferences.

%These principles underlie the
%recommendation algorithms powering websites like Amazon and
%Netflix.\footnote{On Amazon, this is directly evident with statements
%displayed of the form ``users who looked at item X ended up purchasing
%item Y 90\% of the time''.  While the exact inner workings of Netflix
%are not published, the best performing recommendation algorithm in
%the popular Netflix prize competition~\cite{netflix} 
%used an ensemble of CF methods.}

As the web has become more social with the emergence of Facebook,
Twitter, LinkedIn, and most recently Google+, this adds myriad new
dimensions to the recommendation problem by making available a rich
labeled graph structure of social content from which user preferences
can be learned and new recommendations can be made.  In this socially
connected setting, no longer are web users simply described by an IP
address (with perhaps associated geographical information and browsing
history), but rather they are described by a rich user profile (age,
gender, location, educational and work history, preferences, etc.)
and a rich history of user interactions with their friends (comments/posts, 
clicks of the like button, tagging in photos, mutual group
memberships, etc.).  This rich information poses both an amazing
opportunity and a daunting challenge for machine learning methods
applied to social recommendation --- how do we fully exploit rich social
network content in recommendation algorithms?

Many existing \emph{social CF} (SCF)
approaches~\cite{ste,sorec,lla,socinf,sr,rrmf} extend \emph{matrix
factorization} (MF) techniques such as~\cite{pmf} used in the
non-social CF setting.  These MF approaches have proved quite powerful
and indeed, we will show empirically in Section~\ref{sec:EmpResults} 
that existing social extensions of MF outperform a variety of other
non-MF SCF baselines.  The power of CF MF methods stems from their
ability to project users and items into latent vector spaces of
reduced dimensionality where each is effectively grouped by
similarity; in turn, the power of many of the SCF MF extensions stems
from their ability to use social network evidence to further constrain
(or regularize) latent user projections.
%Given the strong performance of existing MF approaches to SCF, we aim
%to further improve on their performance in this paper.  To do this, we
%have first identified a number of major deficiencies of 

Despite providing state-of-the-art performance on SCF problems, we
notice that existing SCF MF objective functions can be improved in
three key aspects, which form the basis for our key algorithmic
contributions in this paper:
\begin{enumerate}
\item[(a)] {\bf Learning user similarity:}
In existing SCF MF objectives, the mapping from user features to user
similarity is fixed.  It will be desirable to learn such similarity
among a large number of profile attributes from data, such as two
users are more similar when they have the same gender or age.
To address this, we extend existing \emph{social regularization} and
\emph{social spectral regularization} objectives to incorporate
\emph{user features} when learning user-user similarities in a
\emph{latent} space.
\item[(b)] {\bf Direct learning of user-to-user information diffusion:}
Existing SCF MF objectives do not permit directly modeling
user-to-user information diffusion according to the social graph
structure.  For example, if a certain user \emph{always} likes content
liked by a friend, this cannot be directly \emph{learned} by
optimizing existing SCF MF objectives.
To address this, we define a new hybrid SCF method where we
\emph{combine} the \emph{collaborative filtering (CF) matrix
factorization (MF) objective} used by Matchbox~\cite{matchbox} with a
\emph{linear content-based filtering (CBF) objective} used to model
direct user-user information diffusion in the social network.
\item[(c)] {\bf Learning restricted interests:} 
Existing SCF MF objectives treat users as globally (dis)similar
although they may only be (dis)similar in specific areas of latent interest.
For example, a friend and their co-worker may both like
technology-oriented news content, but have differing interests when it
comes to politically-oriented news content.
To address this, we define a new social co-preference regularization
method that \emph{learns from pairs of user preferences} over the same
item to learn \emph{user similarities in specific areas} --- a
contrast to previous methods that typically enforce global user
similarity when regularizing.
\end{enumerate}

%We propose to leverage \emph{co-preference 
%(dis)agreement}, i.e., whether two users agreed or disagreed in their
%rating of the \emph{same} item, to encourage \emph{learning}
%latent areas of user preference. %where two users are (dis)similar.
%\end{enumerate}
%This paper proposes novel objective functions to address aspects (a)--(c)
%in a unified latent factorization framework, optimized using gradient descent. 
The key application contribution of our paper is to evaluate the
proposed recommendation algorithms in online human trials of a
custom-developed Facebook App for link recommendation.  We use data
collected over five months from over 100 App users and their 37,000+
friends.  Results show that feature-based social spectral 
regularization outperforms (i) a range of
existing CF and SCF baselines, (ii) performs as well on friend
recommendations as direct modeling of information diffusion features,
and (iii) provides better social regularization than the co-preference
approach.

In addition, deploying our algorithm on a real social network provided
us with a number of interesting observations from user behavior and
feedback discussed in Section~\ref{sec:behavior}.  For example, click
feedback correlates weakly with like ratings.  Also, the most
popular links may be liked by the most people, but they are not liked
by everyone on average.

In the rest of this paper, Section~\ref{sec:Background} provides a
succint overview of CF and SCF algorithms,
Section~\ref{sec:NewObjFuns} proposes three novel objective functions
to address (a)--(c), Section~\ref{sec:Evaluation} dicusses the details
of our Facebook application for link recommendation,
Section~\ref{sec:EmpResults} presents two rounds of evaluation 
with further analysis of user data in our
social recommendation setting, and Section~\ref{sec:Conclusions}
concludes this study.

%All combined, this paper represents a critical step forward in
%powerful SCF recommendation algorithms based on latent factorization
%methods and their ability to fully exploit the breadth of information
%available on social networks.
