%%
%% Template conclusion.tex
%%

\chapter{Conclusion}

The Social Regularization constraint of Social Matchbox which has been detailed in other papers gives a very strong performance as a social recommender. It handily beat the other recommenders in the first trial and this was also reflected in offline experiments when training on the UNION dataset and testing on APP-USER-ACTIVE-ALL, the dataset combination that most closely resembles the training and recommendation data of the live trials. We also found that the Social Regularization constraint performed very differently when recommending friend links and when recommending non-friend links, and this was also reflected in the offline experiments. 

The results of the second trial indicate that the Social Spectral Regularization constrain may be better for social regularization, the two algorithms that used it performed better than the two algorithms that used the Social Regularization constraint. However, this result can't be found in the offline experiments. What can be found in the offline experiments when training on the UNION data was same decrease in performance of the algorithms when recommending non-friend links compared to recommending friend links.

The results of the second trial highlight one difficulty encountered with evaluating the recommendation algorithms: which metric best correlates with human preferences. This paper uses the mean average precision as the main evaluation metric, but there may be other metrics that can better reflect the results of live user trials. Future work on social recommendation that may be crucial can be evaluation of different metric show closely they reflect user preferences.

The social recommenders discussed in this paper try to project the user preferences into the latent space, but it was found that other implementation issues greatly affect the user perceptions of the quality of the links being recommended. One complaint was that since the majority of the LinkR users were mainly English speakers, they automatically disliked links with non-English descriptions and those that pointed to non-English pages. A quick fix we did for this was to just stop recommending non-English links. In the future, having known languages in the user feature and the language of the link or its description in the link feature may result in a lot better and more accurate recommendation, especially for users knowledgeable in non-English languages.

With regards to link recommendations in LinkR, it was found that users were more likely to like recommended links when those links were originally posted by their friends, as opposed to links that were originally posted by non-friends. This was found in both the first and second user trials, and even in the offline experiments. This may be because users will trust a link more when it has been "vouched" for by a friend posting it, rather than just any link from a random stranger.

%%% Local Variables: 
%%% mode: latex
%%% TeX-master: "thesis"
%%% End: 
