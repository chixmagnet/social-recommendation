\chapter{New Algorithms for Social Recommendation}

After the first trial, we made use of the results and what we learned to come up with new algorithms for social recommendation. Again, these new algorithms each form a component of a minimization objective $Obj$ which is composed of sums of one or more objective components:

\begin{align}
\mathit{Obj} = \sum_i \lambda_i \mathit{Obj}_i
\end{align}

Again, a sigmoidal transform 
\begin{align}
\sigma(o) & = \frac{1}{1 + e^{-o}}
\end{align}
of regressor outputs $o \in \R$ is used to squash the outputs 
to the range $[0, 1]$.  
In places where the $\sigma$ transform may be optionally included, 
this is written as $[\sigma]$.  

\section{Hybrid Objective}

Because of the good performance of SVM, we decided to make use of its features $\f_{\x,\y}$  to make another linear regressor. Using $\la \cdot,\cdot \ra$ to denote an inner product, we define a weight
vector $\w \in \R^F$ such that $\la \w, \f_{\x,\y} \ra = \w^T \f_{\x,\y}$ is the prediction of the system. The objective component of linear CBF is therefore

\begin{align}
\sum_{(\x,\y) \in D} \frac{1}{2} (R_{\x,\y} - [\sigma] \w^T \f_{\x,\y})^2
\end{align}

However instead of using this linear CBF model by itself, we combine its predictions with the  Matchbox matrix factorization prediction model $[\sigma] \x^T U^T V y$, to get a hybrid objective component. The full objective component for this hybrid model is

\begin{align}
\sum_{(\x,\y) \in D} \frac{1}{2} (R_{\x,\y} - [\sigma] \w^T \f_{\x,\y} - [\sigma] \x^T U^T V y)^2
\end{align}

\section{New Social Regularizers}

\subsection{Social Spectral Regularization}
\begin{align}
\sum_{\x} & \sum_{\z \in \mathit{friends}(\x)} \frac{1}{2} S^+_{\x,\z} \| U\x - U\z \|_2^2 \nonumber \\
& = \sum_{\x} \sum_{\z \in \mathit{friends}(\x)} \frac{1}{2} S^+_{\x,\z} \| U (\x - \z) \|_2^2 \nonumber \\
& = \sum_{\x} \sum_{\z \in \mathit{friends}(\x)} \frac{1}{2} S^+_{\x,\z} (\x - \z)^T U^T U (\x - \z)
\end{align}

\subfive Note: standard spectral regularization assumes $S^+_{\x,\z} \in [0,1]$;
however we may also want to try $S_{\x,\z}$ since a negative value actively
encourages the latent spaces to oppose each other, which may be desired.

\subsection{Social Co-preference Regularization}

A crucial aspect missing from other SCF methods is that while two users may not be globally similar or opposite 
in their preferences, there may be sub-areas of their interests which can be correlated to each other.
For example, two friends may have similar interests concerning music, but 
different interests concerning politics.  The social co-preference regularizers
aim to learn such selective co-preferences. The motivation is to constrain users $\x$
and $\z$ who have similar or opposing
preferences to be similar or opposite in the same latent latent space
relevant to item $\y$.  

We use $\la \cdot, \cdot \ra_{\bullet}$ to denoe a reweighted inner product. The objective component for 
social co-preference regularization along with its expanded form is

\begin{align}
\sum_{(\x,\z,\y) \in C} & \frac{1}{2} (P_{\x,\z,\y} - \la U\x, U\z \ra_{V\y})^2 \nonumber \\
& = \sum_{(\x,\z,\y) \in C} \frac{1}{2} (P_{\x,\z,\y} - \x^T U^T \diag(V\y) U \z)^2
%= & \sum_{(\x,\z,\y) \in C}  \frac{1}{2} (P_{\x,\z,\y} - \sum_{k=1}^K (U\x)_k (U\z)_k (V\y)_k )^2 
\end{align}


\subsection{Social Co-preference Spectral Regularization}
This is the same as the social co-preference regularization above, except that it uses the spectral regularizer format for 
learning the co-preferences.

 We use $\| \cdot \|_{2,\bullet}$ to denote a re-weighted $L_2$ norm. The objective component for
 social co-preference spectral regularization along with its expanded form is
 
\begin{align}
\sum_{(\x,\z,\y) \in C} & \frac{1}{2} P_{\x,\z,\y} \| U\x - U\z \|_{2,V\y}^2 \nonumber \\
& = \sum_{(\x,\z,\y) \in C} \frac{1}{2} P_{\x,\z,\y} \| U (\x - \z) \|_{2,V\y}^2 \nonumber \\
& = \sum_{(\x,\z,\y) \in C} \frac{1}{2} P_{\x,\z,\y} (\x - \z)^T U^T \diag(V\y) U (\x - \z)
%= & \sum_{\x} \sum_{\z \neq \x} \sum_{\y} \frac{1}{2} P_{\x,\z,\y} \sum_{k=1}^K \big( \left[ (U\x)_k - (U\z)_k \right] (V\y)_k \big)^2
\end{align}

\subsection{Derivatives}
As before, we seek to optimize sums of the above objectives and will use
gradient descent for this purpose. We again use the following useful abbreviations:

\begin{align*}
\s & = U \x \qquad \s_{k} = (U \x)_{k}; \; k=1\ldots K\\
\t & = V \y \qquad \t_{k} = (V \y)_{k}; \; k=1\ldots K
\end{align*}

The derivatives for the linear CBF and hybrid objective functions, as well as the new social regularizers are
 
\begin{itemize}
\item {\bf Explicit Linear CBF}:
\begin{align*}
\frac{\partial}{\partial \w} \Obj_\pcbf & = \frac{\partial}{\partial \w} \sum_{(\x,\y) \in D} \frac{1}{2} \left( \underbrace{(R_{\x,\y} - [\sigma] \overbrace{\w^T \f_{\x,\y}}^{o_{\x,\y}})}_{\delta_{\x,\y}} \right)^2\\
& = \sum_{(\x,\y) \in D} \delta_{\x,\y} \frac{\partial}{\partial \w} - [\sigma] \w^T \f_{\x,\y}\\
& = - \sum_{(\x,\y) \in D} \delta_{\x,\y} [\sigma(o_{\x,\y}) (1 - \sigma(o_{\x,\y}))] \f_{\x,\y}
\end{align*}

\item {\bf Hybrid}:
\begin{align*}
\frac{\partial}{\partial \w} \Obj_\phy & = \frac{\partial}{\partial \w} \sum_{(\x,\y) \in D} \frac{1}{2} \left( \underbrace{R_{\x,\y} - [\sigma] \overbrace{\w^T \f_{\x,\y}}^{o^1_{\x,\y}} - [\sigma] \x^T U^T V\y}_{\delta_{\x,\y}} \right)^2 \\
& = \sum_{(\x,\y) \in D} \delta_{\x,\y} \frac{\partial}{\partial \w} - [\sigma] \w^T \f_{\x,\y} \\
& = - \sum_{(\x,\y) \in D} \delta_{\x,\y} [\sigma(o^1_{\x,\y}) (1 - \sigma(o^1_{\x,\y}))] \f_{\x,\y} 
\end{align*}
\begin{align*}
\frac{\partial}{\partial U} \Obj_\phy & = \frac{\partial}{\partial U} \sum_{(\x,\y) \in D} \frac{1}{2} \left( \underbrace{R_{\x,\y} - [\sigma] \w^T \f_{\x,\y} - [\sigma] \overbrace{\x^T U^T V\y}^{o^2_{\x,\y}}}_{\delta_{\x,\y}}\right)^2 \\
& = \sum_{(\x,\y) \in D} \delta_{\x,\y} \frac{\partial}{\partial U} - [\sigma] \x^T U^T V\y \\
& = - \sum_{(\x,\y) \in D} \delta_{\x,\y} [\sigma(o^2_{\x,\y}) (1 - \sigma(o^2_{\x,\y}))] \t \x^T\\
%\end{align*}
%\begin{align*}
\frac{\partial}{\partial V} \Obj_\phy & = \frac{\partial}{\partial V} \sum_{(\x,\y) \in D} \frac{1}{2} \left( \underbrace{R_{\x,\y} - [\sigma] \w^T \f_{\x,\y} - [\sigma] \overbrace{\x^T U^T V\y}^{o^2_{\x,\y}}}_{\delta_{\x,\y}}\right)^2 \\
& = \sum_{(\x,\y) \in D}  \delta_{\x,\y} \frac{\partial}{\partial V} - [\sigma] \x^T U^T V\y \\
& = - \sum_{(\x,\y) \in D}  \delta_{\x,\y} [\sigma(o^2_{\x,\y}) (1 - \sigma(o^2_{\x,\y}))] \s \y^T \\
\end{align*}

\item {\bf Social spectral regularization}:
\begin{align*}
\frac{\partial}{\partial U} \Obj_\rss & = \frac{\partial}{\partial U} \sum_{\x} \sum_{\z \in \mathit{friends}(\x)} \frac{1}{2} S^+_{\x,\z} (\x - \z)^T U^T U (\x - \z) \\
& = \sum_{\x} \sum_{\z \in \mathit{friends}(\x)} \frac{1}{2} S^+_{\x,\z} U ((\x - \z)(\x - \z)^T + (\x - \z)(\x - \z)^T)\\
& = \sum_{\x} \sum_{\z \in \mathit{friends}(\x)} S^+_{\x,\z} U (\x - \z)(\x - \z)^T
\end{align*}
\end{itemize}
Before we proceed to the final derivatives, we define one additional
vector abbreviation: 
\begin{align*}
\r & = U \z \qquad \r_{k} = (U \z)_{k}; \; k=1\ldots K .
\end{align*}
\begin{itemize}
\item {\bf Social co-preference regularization}:
\begin{align*}
\frac{\partial}{\partial U} \Obj_\rsc & = \frac{\partial}{\partial U} \sum_{(\x,\z,\y) \in C} \frac{1}{2} \left( \underbrace{P_{\x,\z,\y} - \x^T U^T \diag(V\y) U \z}_{\delta_{\x,\z,\y}} \right)^2\\
& = \sum_{(\x,\z,\y) \in C} \delta_{\x,\z,\y} \frac{\partial}{\partial U} - \x^T U^T \diag(V\y) U \z \\
%%%%%%%%%%%%%%%%%%%%%%%%%%%%%%%%%%%%%%%%%%%%%%%%%%%%%%%%%%%%%%%%%%%%%%%%
%& = \delta \frac{\partial}{\partial U} - \tr(\diag(\x) U^T \diag(V\y) U \diag(\z)) \\
%& = - \delta \diag(\z) \diag(\x) U^T \diag(V\y) + \diag(\x)^T \diag(\z)^T U^T \diag(V\y)^T\\
%& = - \delta \diag(V\y)^T U \diag(\x)^T \diag(\z)^T + \diag(V\y)^T U \diag(\z)^T \diag(\x)^T\\
%& = - \delta \diag(V\y)^T U (\diag(\x) \diag(\z) + \diag(\z) \diag(\x)) \\
%& = - \delta \diag(V\y)^T U (\z \x^T + \x \z^T) \\
%%%%%%%%%%%%%%%%%%%%%%%%%%%%%%%%%%%%%%%%%%%%%%%%%%%%%%%%%%%%%%%%%%%%%%%%
% Found it, see here for direct derivative: http://www.ee.ic.ac.uk/hp/staff/dmb/matrix/calculus.html
& = - \sum_{(\x,\z,\y) \in C} \delta_{\x,\z,\y} (\diag(V\y)^T U \x \z^T + \diag(V\y) U \z \x^T)\\ % \diag(V\y)^T = \diag(V\y)
& = - \sum_{(\x,\z,\y) \in C} \delta_{\x,\z,\y} \diag(V\y) U (\x \z^T + \z \x^T)\\
\end{align*}
In the following, $\circ$ is the Hadamard elementwise product:
\begin{align*}
\frac{\partial}{\partial V} \Obj_\rsc & = \frac{\partial}{\partial V} \sum_{(\x,\z,\y) \in C} \frac{1}{2} (P_{\x,\z,\y} - \x^T U^T \diag(V\y) U \z)^2\\
 & = \frac{\partial}{\partial V} \sum_{(\x,\z,\y) \in C} \frac{1}{2} \left( \underbrace{P_{\x,\z,\y} -  (\overbrace{U\x}^\s \circ \overbrace{U\z}^\r)^T V\y}_{\delta_{\x,\z,\y}} \right)^2\\
 & = \sum_{(\x,\z,\y) \in C} \delta_{\x,\z,\y} \frac{\partial}{\partial V} - (\s \circ \r)^T V\y\\
 & = - \sum_{(\x,\z,\y) \in C} \delta_{\x,\z,\y} (\s \circ \r) \y^T
\end{align*}
\item {\bf Social co-preference spectral regularization}:
\begin{align*}
\frac{\partial}{\partial U} \Obj_\rscs & = \frac{\partial}{\partial U} \sum_{(\x,\z,\y) \in C} \frac{1}{2} P_{\x,\z,\y} (\x - \z)^T U^T \diag(V\y) U (\x - \z)\\
& = \sum_{(\x,\z,\y) \in C} \frac{1}{2} P_{\x,\z,\y} \left( \diag(V\y)^T U (\x - \z) (\x - \z)^T \right.\\
& \left. \qquad \qquad \qquad \qquad + \diag(V\y) U (\x - \z) (\x - \z)^T \right)\\
& = \sum_{(\x,\z,\y) \in C} P_{\x,\z,\y} \diag(V\y) U (\x - \z) (\x - \z)^T\\
\frac{\partial}{\partial V} \Obj_\rscs & = \frac{\partial}{\partial V} \sum_{(\x,\z,\y) \in C} \frac{1}{2} P_{\x,\z,\y} (\x - \z)^T U^T \diag(V\y) U (\x - \z)\\
& = \frac{\partial}{\partial V} \sum_{(\x,\z,\y) \in C} \frac{1}{2} P_{\x,\z,\y} (U(\x-\z) \circ U(\x-\z))^T V\y\\
& = \frac{1}{2} \sum_{(\x,\z,\y) \in C} P_{\x,\z,\y} (U(\x-\z) \circ U(\x-\z)) \y^T
\end{align*}
\end{itemize}

Hence, for any choice of primary objective and one or more regularizers,
we simply add the derivatives for each of $\w$, $U$, and $V$
according to~\eqref{eq:sum_der}.

\section{Results}

\subsection{Passive Results}

\subsection{LinkR Results}

