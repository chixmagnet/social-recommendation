%%
%% Template conclusion.tex
%%

\chapter{Model Combinations}
\label{cha:bma}

In this section we discuss two different approaches to combining the information we learned above.

\section{Positive Feature Selection}
\label{sec:notation}

Firstly, we can combine features which were successful at improving from our baseline.

Using the combined feature set of:
\begin{itemize}
\item Traits
\item Pages
\item Groups
\end{itemize}

\begin{figure}[h]
	\begin{center}
		\includegraphics[scale=0.75]{results/combination/bar_combination.pdf}
		\caption{Predictors paradigm}
	\end{center}
\end{figure}

\begin{figure}[h]
	\begin{center}
		\includegraphics[scale=0.75]{results/combination/line_combination.pdf}
		\caption{Predictors paradigm}
	\end{center}
\end{figure}


\section{Bayesian Model Averaging}
\label{sec:bma}

%%% Local Variables: 
%%% mode: latex
%%% TeX-master: "thesis"
%%% End: 
