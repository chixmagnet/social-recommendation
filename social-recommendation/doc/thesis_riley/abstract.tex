%%
%% Template abstract.tex
%%

\chapter*{Abstract}
\label{cha:abstract}
\addcontentsline{toc}{chapter}{Abstract}

Social networks provide a wide array of user specific interactions, profile information and user preferences.
This thesis attempts to decipher which user interactions or preferences are truly indicative of 'likes', this 
information is then leveraged to allow for binary classification of user specific links with the goal of discovering the 
ideal combination of information for prediction.

The success of our predictions are evaluated using a number of machine learning algorithms including, 
\emph{Naive Bayes} (NB), \emph{Logistic Regression} LR and \emph{Support Vector Machines} (SVM), results are compared to previous 
work using \emph{Matchboxing} (MB) and \emph{Social Matchboxing} (SMB) techniques as baselines. The data set is sourced from a 
set of over 100 Facebook users and their interactions with over 30,000 friends during a four month period.

Our analysis has shown that user interactions in themselves are not predictive of user likes, while user preferences are. These 
results, coupled with user like exposure curves offer a useful paradigm for extracting and exploiting user preferences for prediction 
across our data set.

%%% Local Variables: 
%%% mode: latex
%%% TeX-master: "thesis"
%%% End: 
